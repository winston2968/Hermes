\documentclass{article}


\setlength{\parindent}{0cm}


\input{~/Documents/Licence/LaTeX_Models/HeaderTest.tex}


\begin{document}


% ==================================================================================================================================


\begin{center}
    \textbf{ - Projet Réseau - }

    \Large{\textbf{Hermes}}
\end{center}

\rule{\linewidth}{1.5pt}


% ==================================================================================================================================

\justify


\subsection{Introduction}

\section{Chatty}

\subsection{Principe}

Chatty est une application de chat en ligne de commande permettant à deux utilisateurs de discuter en toute sécurité grâce au protocole RSA. 

Un utilisateur se connecte en tant que serveur, l'autre en tant que client. 
Le serveur doit physiquement donner son adresse IP au client et le client doit la rentrer dans le logiciel pour 
permettre la connexion. Ici, on utilise des sockets pour établir une connexion. La connexion par socket n'étant pas chiffré, 
on utilise un système de chiffrement RSA pour, d'une part permettre l'échange des clés en toute sécurité, et d'autre part, 
une communication sécurisée. 




\section{Hermes}












\end{document}
